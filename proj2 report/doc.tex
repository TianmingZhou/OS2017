\documentclass{article}
\usepackage{amsmath,amsfonts,amsthm,amssymb}
\usepackage{fancyhdr}
\usepackage{extramarks}
\usepackage{chngpage}
\usepackage{color}
\usepackage{graphicx,float}

\title{OS Project 2 Documentation}
%\date{}
\author{Tianming Zhou\ 2014011393}

% In case you need to adjust margins:
\topmargin=-0.45in      %
\evensidemargin=0in     %
\oddsidemargin=0in      %
\textwidth=6.5in        %
\textheight=9.0in       %
\headsep=0.25in         %

\begin{document}
\maketitle \thispagestyle{empty}

\section{Task 1: System Call -- File And }

\subsection{Brief Description of Implementation}

\begin{itemize}
\item See section 3 for \texttt{halt}. See section 2 for address translation.
\item 
a non-\texttt{static} bitmap array \texttt{filedescriptor\_list} of type \texttt{OpenFile}, size 16, in class \texttt{UserProcess}:
\begin{itemize}
\item
This array is not \texttt{static}.
\item
\texttt{filedescriptor\_list[i]} is non-null if and only if the file descriptor \texttt i has been allocated.
\item
Initially, 0 and 1 are allocated for \texttt{stdin} and \texttt{stdout}, using \texttt{UserKernel.console.openForReading()} and \texttt{UserKernel.console.openForWriting()}, respectively.
The other 14 descriptors are set to \texttt{null}.
\item
The array is initialized in the constructor of class \texttt{UserProcess}.
When the process is terminating, an \texttt{exit} is thrown and thus its handler is called.
Then the handler call \texttt{unloadSections}, which calls \texttt{close}'s handler to close the files.
\item
When we request for a valid file descriptor for a file, we call \texttt{consumeFileDescriptor (OpenFile)} with the file as argument.
The function finds the smallest idle file descriptor, say \texttt i, by traversing \texttt{filedescriptor\_list} with indices from 2 to 15.
It sets \texttt{filedescriptor\_list[i]} to be the only argument and returns \texttt i.
If all of the 14 descriptors have been allocated, this function returns -1.
\item
When a descriptor should be recycled, we call \texttt{releaseFileDescriptor (int)}.
This function tests if the descriptor is within range from 0 to 15 and does nothing if not.
If the descriptor is valid, no matter whether it is being used,  \texttt{filedescriptor\_list[i]} will be set to \texttt{null}.
\item
When we'd like to get the file corresponding to descriptor \texttt i, we call \texttt{getOpenFilefromFileDescriptor (int)}.
This function returns \texttt{filedescriptor\_list[i]}, if \texttt i is valid, and does nothing otherwise.
This function may returns \texttt{null}, which indicates this descriptor has not been associated with a file.
\end{itemize}
\item
We maintain the status whether a file (or filename) is still open in some process, using class \texttt{UserProcess.FileReference}.
	\begin{itemize}
	\item
	This class is \texttt{static}.
	\item
	Each object of this class has two variable: \texttt{cnt} is the number of processes that have it open, and \texttt{dlt} is true if and only if the corresponding file is to be removed.
	\item
	It is this class's responsibility to ask file system to delete files.
	\item
	When any function in this class is running, interrupt is disabled.
	\item
	This class maintain a \texttt{static} map from strings (filenames) to \texttt{FileReference} objects.
	A filename is in this map if and only if some process has it open.
	\item
	The function \texttt{fetchFR()} is used to extract the \texttt{FileReference} for the given filename.
	If the filename is not in the map, it maps the string to a new \texttt{FileReference} object with both variables 0 (false).
	\item
	The function \texttt{tryRemove} removes the filename from the map if the corresponding \texttt{cnt} is zero, and removes the file by calling \texttt{UserKernel.fileSystem.remove}, if \texttt{cnt} is zero and \texttt{dlt} is true.
	This function returns -1 if it fails, 1 if the file is deleted immediately, and 0 otherwise.
	A return $\geq0$ is considered as success.
	\item
	When a file is referred somewhere, function \texttt{refer(String)} should be called.
	Firstly, this function fetchs the \texttt{FileReference} object, and then it increases \texttt{cnt} by 1 if \texttt{dlt} is false.
	This function returns the negation of \texttt{dlt}.
	\item
	When a file reference is closed, function \texttt{derefer(String)} should be called.
	This function fetchs the \texttt{FileReference} object, decreases \texttt{cnt} by 1, and calls \texttt{tryRemove}.
	A negative \texttt{cnt} throws an internal exception.
	\item
	When we want to delete a file, we call \texttt{unlink}.
	This function fetchs the  \texttt{FileReference} object, set \texttt{dlt} to true, and calls \texttt{tryRemove}.
	\end{itemize}
\item
We placed all system call handler in class \texttt{UserProcess}. See \texttt{handleSyscall} for their names.
\item
We uses \texttt{UserKernel.fileSystem} as the file system.
\item \texttt{creat} and \texttt{open}:
\begin{itemize}
\item
We handle the two system calls together.
This handler returns -1 if one of the following conditions is satisfied:
\begin{itemize}
\item
The input address is negative;
\item
\texttt{readVirtualMemoryString} returns \texttt{null}. This happens if the filename is empty string, if the filename is too long, and if the address is too large, etc;
\item
The file is to be deleted, i.e., \texttt{FileReference.refer (filename)} returns true;
\item
The file can not be opened by \texttt{UserKernel.fileSystem.open ()}, i.e., it returns a null \texttt{OpenFile};
\item
We are unable to find an idle file descriptor within range from 2 to 15.
\end{itemize}
\item
When we fail, rollback may be necessary: calls to \texttt{FileReference.derefer (filename)} and \texttt{OpenFile.close()}.
\item
A pitfall is that the original file in the disk is overrided when the second argument of \texttt{UserKernel.fileSystem.open ()} is true.
Therefore, we first test opening the file with the argument false.
If we fail, we try again with it true.
If we also fail, we do fail.
\end{itemize}
\item
\texttt{read}:
\begin{itemize}
\item
This handler returns -1 if the given address is minus or if the descriptor is invalid or idle.
Reading from file less than \texttt{count} bytes does \emph{not} cause this to return -1, but writing less bytes than \emph{reading} (instead of \texttt{count}) to buffer \emph{is} an error.
\item
We extract the \texttt{OpenFile} object, read data from the file to an internal buffer by \texttt{OpenFile.read()}, and write it to input buffer by \texttt{writeVirtualMemory}.
\item
We return the value \texttt{writeVirtualMemory} returns directly.
\end{itemize}
\item
\texttt{write}
\begin{itemize}
\item
This handler returns -1 if the given address is minus, if the descriptor is invalid or idle, and if we can't read enough bytes from buffer or we can't write enough bytes to file.
\item
We extract the \texttt{OpenFile} object, read data to an internal buffer by \texttt{readVirtualMemory} and write it to file by \texttt{OpenFile.write()}.
\item
We return the argument \texttt{count} if both \texttt{readVirtualMemory} and  \texttt{OpenFile.write()} return that value, and -1 otherwise.
If both \texttt{readVirtualMemory} does not return \texttt{count}, \texttt{OpenFile.write()} will not be called.
\end{itemize}
\item
\texttt{close}
\begin{itemize}
\item
This handler returns -1 if the descriptor is invalid or idle.
\item
We extract the \texttt{OpenFile} object, store its filename, call \texttt{OpenFile.close()} to close it, recycle the file descriptor.
If the descriptor is not less than 2, we call \texttt{FileReference.derefer()} to maintain the records.
\item
We'd better store the filename before the file is closed, because it may be not available otherwise.
\end{itemize}
\item
\texttt{unlink}
\begin{itemize}
\item
This handler returns -1 if the filename is null, or if the file does not exist, etc.
Calling this multiple times before all reference to this file are closed is equivalent to calling once.
\item
We just call \texttt{FileReference.unlink()} and return -1 if and only if it returns -1.
\end{itemize}
\end{itemize}

\subsection{Test Cases}

\begin{itemize}
\item
\texttt{UserProcess.selfTest\_Ala} tests:
	\begin{itemize}
	\item
	release a file descriptor;
	\item
	release a file descriptor after we have release it successfully;
	\item
	request a file descriptor when descriptor 0 or 1 is closed;
	\end{itemize}

\item
\texttt{test/test\_syscall5} tests closing \texttt{stdout}.

\item
\texttt{test/test\_syscall6}:
	\begin{itemize}
	\item
	write to invalid (-1, 16), unallocated (5) file descriptors, to \texttt{stdin}, \texttt{stdout};
	\item
	creat 20 files with closing any;
	\item
	close files twice;
	\item
	creat, write and close file; open, read and close the same file;
	\item
	creat an existing file, write to and read from it, and close it;
	open it and read it; (The file is partially overrided and extended to the end.)
	\item
	read using a string \texttt{"string"} as buffer;
	\item
	read with invalid (-1, 16), closed file descriptor, from \texttt{stdin};
	\item
	close \texttt{stdin} and then open a file;
	\item
	unlink non-existing file, closed file, deleted file;
	\item
	unlink, by multiple times, file that is opened twice, file that is opened twice and then closed once, file that is opened twice and then closed once;
	creat and open file whose deletion is pending;
	\item
	two processe open one file concurrently, and use same descriptor;
	unlink a file that another process has open;
	unlink a file that has been opened in two processes;
	\item
	creat and unlink a file with a long name;
	\item
	close \texttt{stdout} and then write to it.
	\end{itemize}
\end{itemize}

\section{Task 2: Address Translation}

\subsection{Brief Description of Implementation}

\begin{itemize}
\item
Similar to file descriptor, we maintain a \texttt{static} bitmap array \texttt{pageIdle} of type \texttt{boolean}, size \texttt{Machine.processor().getNumPhysPages()}, in class \texttt{UserKernel}:
	\begin{itemize}
	\item
	This array is \texttt{static}.
	\item
	\texttt{pageIdle[i]} is true if and only it is not idle, i.e., it has been allocated for some process.
	\item
	The static constructor of \texttt{UserKernel} (see code) calls \texttt{initializePages} to initialize every element in this array to be true.
	\item
	When any function in the following is running, interrupt is disabled.
	\item
	\texttt{allocatePage()} should be called when we ask for free pages.
	This function find the free page with the largest index, by traversing the array backwards. Then it set the corresponding element to \texttt{false}.
	If there is no free pages, this function returns -1.
	\item
	\texttt{deallocatePage()} should be called when a page should be recycled.
	This function set the corresponding element in the array to \texttt{true}.
	If the given page id is out of range, an internal error is thrown.
	\item
	\texttt{getPageStates()} returns the status of the page, i.e., the corresponding element directly.
	This function is for debug and test.
	\end{itemize}
\item
For each process, we maintain a array \texttt{pageTable} of type \texttt{translationEntry}, size \texttt{UserProcess.numPages} to map virtual page number to physical page number:
	\begin{itemize}
	\item
	This array is not \texttt{static}.
	\item
	This array is created and initialized in the function \texttt{loadSections()}.
	We ask for pages one by one, using \texttt{UserKernel.allocatePage()}.
	When physical page are insufficient, we get a -1, and then we call \texttt{unloadSections()} to release the pages allocated to the process.
	\item
	In the argument list of \texttt{section.loadPage()}, we replace \texttt{vpn} by \texttt{pageTable[vpn].ppn}.
	\item
	Just after that, we set \texttt{pageTable[vpn].readOnly} to \texttt{true} if the section if read-only.
	\item
	When the process is terminating, \texttt{unloadSections()} is also called to release pages, similar to closing the files above.
	\end{itemize}
\item
\texttt{readVirtualMemory}:
	\begin{itemize}
	\item
	This function reads data page by page.
	For each page, it calculates the virtual page number, the index of the first byte in the page to read and the length in the page.
	The index of the first byte equals to \texttt{vaddr} module \texttt{Processor.pageSize}.
	The length is the smaller one between the number of bytes to read and the number of left bytes in the page.
	\item
	The \texttt{pageTable[i].used} is set to \texttt{true} for the related pages.
	\item
	When the virtual page number is negative or too large, and when the valid bit of the page table is false, we skip the read operations on this page.
	As a result, the content in corresponding buffer is unchanged.
	\end{itemize}
\item
\texttt{writeVirtualMemory} is very similar to \texttt{readVirtualMemory}, the differences are:
	\begin{itemize}
	\item
	In addition, we skip write operations when the page table entry's readOnly bit is true.
	\item
	In addition, the dirty bits of the page table entries are set to true.
	\item
	The argument order to \texttt{System.arraycopy()} is partially reversed.
	\end{itemize}
\end{itemize}

\subsection{Test Cases}

\begin{itemize}
\item
\texttt{UserKernel.selfTest\_Page} tests page allocation and deallocation (recycle).
We maintain the set of pages we own.
Each round, we request a page with .5 probability and release a page in the set with .5 probability.
None of the following should occur when we request for a page:
	\begin{itemize}
	\item
	we obtain an invalid page number;
	we obtain a page number we have already own;
	we obtain -1 when we do not own all pages;
	\end{itemize}
We count the times we request for pages when we own all pages, which is nonzero with high probability.
We count the times we request for 10 pages contiguously to simulate allocate pages for process.
We count the times we release 10 pages contiguously to simulate allocate pages for process. (This does not know which pages belong to the same process.)

\texttt{UserProcess.selfTest\_Cys} tests:
	\begin{itemize}
	\item
	write to the whole memory;
	\item
	write to the memory starting in the middle of one page, through a whole page, and ending at the first byte of the third page;
	\item
	write to a whole page;
	\item
	write to the last byte of one page and the first byte of the next page;
	\item
	read the whole memory with several pages read-only;
	\item
	write a little data to the memory covering position -1 and 0;
	\item
	read a little data from the memory covering position -1 and 0;
	\item
	write a lot of data to the memory covering position -1 and 0;
	\item
	read a lot of data from the memory covering position -1 and 0;
	\item
	write to the whole memory with several pages in the middle read-only;
	\item
	write to the memory with several pages read-only in the middle / at the beginning / in the end;
	\item
	read string from single / multiple page(s);
	\item
	read string in the end of the memory with the last byte in the memory being zero / nonzero.
	\item
	We also test the \texttt{readOnly, used, dirty} bits after each operations and then reset them.
	\end{itemize}

\item
\texttt{test/test\_readonly} tests:
	\begin{itemize}
	\item
	read data from defined / undefined address (one or no variable is declared to occupy this address);
	\item
	write data to legal (defined /undefined) address;
	\item
	write data to illegal address, including negative address, address divided by 4 not evenly, address exceeding upper limit, and their combination.
	\end{itemize}

\item
\texttt{test/test\_syscall4} tests recycling pages, redoing allocation when pages are insufficient, and allocating incontiguous physical pages
\end{itemize}

\section{Task 3: System Call -- Process}

\subsection{Brief Description of Implementation}

\begin{itemize}
\item
\texttt{halt}, globally unique process ID, and identity of root process:
	\begin{itemize}
	\item
	We define a \texttt{final static} variable \texttt{processID4root} to 0.
	A process is the root process if and only if its ID is this number.
	\item
	We define a \texttt{static} variable \texttt{ID\_next} and initialize it to \texttt{processID4root}.
	Each time the constructor of \texttt{UserProcess} is called, the new process's id is set to \texttt{ID\_next}, and then \texttt{ID\_next} is increased by 1.
	\item
	It is clear that every process has a nonnegative ID.
	\item
	By calling \texttt{isRootProcess()}, every process is able to know whether itself is the root, based on the two variables.
	\item
	When the handler is called, it tests whether it is the root.
	If it is, it calls \texttt{Machine.halt()}. If not, it returns immediately.
	\end{itemize}
\item
\texttt{exit} and identity of the last process:
	\begin{itemize}
	\item
	We define a set \texttt{aliveProcess} containing \texttt{Integer} to be the set of existing processes' IDs.
	\item
	In the constructer of \texttt{UserProcess}, the new process's ID is added to the set.
	In the function \texttt{unloadSections()}, the ID is removed from the set.
	\item
	A process is the last process, when terminating, if and only if the set is empty after its ID is removed.
	\item
	The handler first calls \texttt{unloadSections()}, and then checks whether it is the last process.
	It calls \texttt{Kernel.kernel.terminate()} if it is the last one, and \texttt{KThread.finish()} otherwise.
	\item
	When a process is determined to terminate, this handler will be called, no matter whether it terminates normally or an exception is not handled (argument is -1 in this case).
	\end{itemize}
\item
\texttt{exec}
	\begin{itemize}
	\item
	We do a lot of tests here: the addresses should be valid and we can read a string starting from each of them, etc
	\item
	For the $i$-th argument, the address is \texttt{argv+i*4}.
	We first read 4 bytes starting from it, and then read a string from the address returned by \texttt{Lib.bytesToInt()}.
	\item
	Then we call \texttt{UserProcess.newUserProcess()} to obtain a new object, and call its \texttt{execute()}.
	If \texttt{execute()} returns \texttt{true}, we add its ID to \texttt{mapID2UP} (see below), return its ID.
	Else we remove its ID from \texttt{aliveProcess} and return -1.
	\end{itemize}
\item
\texttt{join}, child process, and status:
	\begin{itemize}
	\item
	We define a map \texttt{mapID2UP} from process ID to \texttt{UserProcess} object for each process.
	A process ID is in this set if and only if the process is the current process's child.
	\item
	This map is initialized to empty in the constructor of \texttt{UserProcess}.
	\item
	We use a variable \texttt{thread} of type \texttt{UThread} to store the associated \texttt{UThread} object.
	This variable is initialized in \texttt{execute()}.
	\item
	We use a variable \texttt{status} to store the return value of this process.
	The variable is initialized to a number that is unreasonable for other people, in the constructor of \texttt{UserProcess}.
	\item
	The handler returns -1 if the \texttt{processID} is negative, if the address for \texttt{status} is negative, and if the corresponding process is not the current process's child.
	However, when \texttt{status} points to a read-only region or is too large, the handler returns 0 and the return value of the child process is lost.
	\item
	If all tests are passed, the entry is removed from \texttt{mapID2UP} and the child's thread's \texttt{KThread.join()} is called.
	We return 1 if the child's status is not its initial value and we write 4 bytes to \texttt{status}, and 0 otherwise.
	\item
	In the second time we join to a process, we will find that the process ID is not in the map and thus return -1.
	\end{itemize}
\end{itemize}

\subsection{Test Cases}

\begin{itemize}
\item
\texttt{test/test\_syscall1}
	\begin{itemize}
	\item
	join to self, -1, unused id;
	\item
	join to child's child;
	\item
	join to child;
	\item
	exec with multiple arguments, invalid arguments;
	\item
	compete to print;
	\item
	exec with non-existing file, file with filename not ending with \texttt{.coff};
	\item
	exec with \texttt{status} not divided by 4 evenly, with large \texttt{status}, with \texttt{status} being -1;
	\item
	exec with \texttt{argc=-1}, \texttt{argv=-1};
	\item
	exec a file whose name is too long;
	\item
	exec with an argument that is too long.
	\end{itemize}


\item
\texttt{test/test\_syscall2} tests calling \texttt{halt} from non-root process and root process.

\item
\texttt{test/test\_syscall3} tests calling \texttt{exit} from root, first process and non-root, last process.

\end{itemize}

\section{Task 4: Lottery Scheduler}

\subsection{Brief Description of Implementation}

\begin{itemize}
\item
Both prioirty and ticket refer to ticker here.
\item
We adopt identical architecture.
Unfortunately, we modified \texttt{Queue} and \texttt{ThreadState}, so we have to override all functions.
\item
\texttt{priorityMinimum} is set to 1, and \texttt{priorityMaximum} is set to \texttt{Integer.MAX\_VALUE}.
\item
To maintain tickets, we create in each \texttt{LotteryQueue} and \texttt{ThreadState} a \texttt{SumSet} that extends \texttt{HashMap}:
	\begin{itemize}
	\item
	We use a variable \texttt{sum} to maintain the sum of all elements' sum in this set.
	\item
	We override functions \texttt{add()} and \texttt{remove()}.
	When any of the two functions is running, interrupt is disabled.
	\item
	When an object (a \texttt{LotteryQueue} or a \texttt{ThreadState}) is added or removed, we call proper function (\texttt{getDonatingPriority()} or \texttt{getEffectivePriority()}) to obtain the difference and modify \texttt{sum} by this amount.
	\end{itemize}
\item
Deadlock:
	\begin{itemize}
	\item
	When a deadlock forms, the \texttt{SumSet} will fall into an infinite loop.
	\item
	We use three \texttt{static} variables \texttt{LotteryQueue loop\_mark} and \texttt{int loop\_cnt, loop\_k} to identify such loops.
	All are initialized to zero (\texttt{null}).
	\item
	\texttt{loop\_mark} is \texttt{null} if and only if we are not updating priority.
	\item
	The algorithm is simple.
	Imagine we are walking in directed graph where the out-degree is at most 1 for all vertices, and we can only memorize two vertices, including our current vertex.
	Besides, we have two counters.
	In round $k=0,1,\ldots$, we walk $2^k$ steps and then memorize the current vertex.
	After each step, we examine whether our current vertex is exactly the one in our memory.
	If it is, we find a loop.
	If the current vertex has not out-edge, we complete.
	This algorithm terminates when $2^k$ is not less than the length of the only loop.
	\item
	When we update priority, we call \texttt{getDonatingPriority()} and \texttt{getEffectivePriority()} alternatively.
	Let each \texttt{LotteryQueue} and \texttt{ThreadState} correspond a vertex.
	If a objext transfers its tickets to another objext, form an edge from the former vertex to the latter vertex.
	Then the out-degree is at most 1.
	\item
	We implement this algorithm on the graph constructed above in \texttt{LotteryScheduler.updateDonatingPriority}.
	\item
	When we begin to update priority, we set \texttt{loop\_mark} to the first \texttt{LotteryQueue} and set \texttt{loop\_cnt=0, loop\_k=1}.
	This should happen if and only if we find \texttt{loop\_mark} is \texttt{null}, after we are sure we have to do update.
	\item
	When we find \texttt{loop\_mark} is exactly the current \texttt{LotteryQueue}, we are sure that we are in a cycle and we return immediately, setting \texttt{loop\_mark} to \texttt{null}.
	\item
	\texttt{loop\_cnt, loop\_k} are the number of steps we have walked in this round and the number of steps we should walk in this round.
	If the two is the same, the round is finished and we should update the variables.
	\item
	When we find its end, we begin to return and reset \texttt{loop\_mark} to \texttt{null}.
	\item
	See code for detail. The code is really short.
	\end{itemize}
\item
\texttt{LotteryQueue.pickNextThread}
	\begin{itemize}
	\item
	We call \texttt{waitingQueue.getSum ()} to obtain the total number of tickets and then ask for a random number according to this number.
	We traverse each process in \texttt{waitingQueue} by its iterator, and decrease the random number by the number of tickets the process owned.
	We return the first process that makes the random number negative.
	\end{itemize}
\item
\texttt{LotteryQueue.updateDonatingPriority}
	\begin{itemize}
	\item
	A queue with \texttt{transferPriority} false donates zero tickets instead of \texttt{priorityMaximum}, so we return immediately if it is false.
	\item
	The new donating priority is obtained by calling \texttt{waitingQueue.getSum()}.
	\end{itemize}
\item
\texttt{ThreadState.updateEffectivePriority}
	\begin{itemize}
	\item
	The new priority is the sum of current process's internal ticket and \texttt{acquires.getSum()}.
	\end{itemize}
\end{itemize}

\subsection{Test Cases}

\begin{itemize}
\item
\texttt{LotteryScheduler.endoderm()} tests the join operation and ticket transfer via `join queue'.
\item
\texttt{LotteryScheduler.mesoderm()} tests \texttt{setPriority()} before exection (\texttt{fork()}).
\item
\texttt{LotteryScheduler.ectoderm()} tests \texttt{increasePriority()} and \texttt{decreasePriority()} during execution.
\item
\texttt{LotteryScheduler.methylcyclopropane()} tests deadlock.
Namely, three threads form a cycle of size 3, with high probability.
And then the fourth thread join one of the former 3 threads.
This should not cause scheduler to fall into infinite loop and not to respond.
\end{itemize}

\label{LastPage}
\end{document}
